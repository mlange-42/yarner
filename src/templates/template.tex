\documentclass[11pt,a4paper]{article}

\usepackage{listings}
\usepackage{color}

\definecolor{comment}{rgb}{0.6, 0.6, 0.6}
\definecolor{string}{rgb}{0.6, 0.26, 0.26}

\lstnewenvironment{code}[1][]
{\lstset{%
    basicstyle=\small\ttfamily,%
    flexiblecolumns=false,%
    basewidth={0.5em,0.45em},%
    keywordstyle=\color{blue},%
    commentstyle=\color{comment},%
    stringstyle=\color{string},%
    breaklines=true,%
    frame=tb,%
    aboveskip=1cm,%
    belowskip=1cm,%
    showstringspaces=false,%
    literate={<=}{{$\le$}}2 {>=}{{$\ge$}}2 {!=}{{$\ne$}}2%
    {=>}{{$\Rightarrow$}}2 {==>}{{$\Longrightarrow$}}2%
    {->}{{$\rightarrow$}}2 {<-}{{$\leftarrow$}}2%
    {\ .}{{$\circ$}}2 {\ .\ }{{$\circ$}}2,%
    title=\lstname,%
    #1%
}}
{}

\title{TeX outline template}

\begin{document}

\maketitle

The following code goes to the base file of code output:

\begin{code}
Hello Literate Programmer!
==> More code.
\end{code}

In the code output, \verb`==> More code.` will be replaced by the following code:

\begin{code}[name=More code]
Have fun with outline!
\end{code}

To create code in other files, use \verb`file:<path/to/file>` as block name.
Here, we create a file \verb`main.rs` in subfolder \verb`src`:

\begin{code}[language=rust,name=file:src/main.rs]
fn main() {
    println!("Hello Literate Programmer!");
    ==> More code in main.
}
\end{code}

Pulling code together works as usual:

\begin{code}[language=rust,name=More code in main]
println!("Have fun with outline!");
\end{code}
\end{document}
